%!TEX root = tese.tex
\chapter{Conclusão e perspetivas de trabalho futuro} % (fold)
\label{cha:conc}
No trabalho publicado em 2003 por Eon-Duval \et\ \cite{duvaltff}, no âmbito da utilização de uma operação de ultrafiltração no isolamento intermediário de DNA plasmídico, é possível ler o seguinte:
\begin{quote}
``\ldots As membranas de ultrafiltração utilizadas foram desenvolvidas para a purificação de proteínas e o seu ``cut-off'' determinado pela retenção de proteínas globulares. Contrariamente a proteínas globulares, o DNA plasmídico é uma molécula longa e fina, o que dificulta a previsão do seu comportamento em operações de ultrafiltração. Apesar do tamanho médio de plasmídeos para fins terapêuticos ser de vários milhões de Daltons, o tamanho de poro selecionado neste estudo para reter o plasmídeo é consideravelmente inferior \ldots'' 
\end{quote}
De facto, tal como afirmado por Eon-Duval \et, a permeação de plasmídeos em membranas de micro e ultrafiltração revela um comportamento completamente distinto do caso de proteínas globulares, que essencialmente se podem modelar como esferas rígidas e para as quais existem modelos bem estabelecidos na literatura. No caso do pDNA, é um facto bem conhecido que este exibe valores de permeação significativos em membranas com poros de tamanho consideravelmente inferiores. Este comportamento impede assim a aplicação dos modelos de esferas rígidas para prever a permeação de pDNA em membranas de micro e ultrafiltração. Na presente tese de doutoramento procurou-se superar esta limitação pelo desenvolvimento de novos modelos que possam contabilizar efeitos da deformação molecular induzida pelas tensões de corte geradas à entrada dos poros, e assim permitir explicar este fenómeno. O conceito de deformação molecular foi igualmente explorado por outros autores. No entanto, apesar de o trabalho desenvolvido ser de inquestionável valor e de ter proporcionado importantes avanços no estado do conhecimento nesta matéria, os modelos desenvolvidos apresentam algumas desvantagens que podem ser enumeradas, e que foram previamente discutidas na secção~\ref{sec:modulaçãopermeação}. 

Com a realização do presente trabalho foi possível obter uma ampliação do espectro de aplicação do modelo do transporte restringido, provavelmente o modelo mais citado e utilizado no âmbito das operações de micro e ultrafiltração. Com esta nova abordagem passa a ser possível aplicar o modelo, não só a moléculas com uma geometria rígida, como também a moléculas que exibem elevados graus de flexibilidade e como consequência valores significativos de permeação em poros com tamanhos inferiores. As moléculas foram modeladas como cadeias FJC e CSC e os seus coeficientes de partição determinados por simulações estocásticas. Nestas simulações determina-se a distribuição radial da probabilidade de uma molécula entrar no poro estabelecendo uma condição de entrada. No presente trabalho considerou-se como condição de entrada que a massa pontual mais perto do poro deverá estar projetada no interior do mesmo. Foi possível determinar uma equação que relaciona o coeficiente de partição com \lambdag, e verificou-se que a equação é válida tanto para a abordagem FJC, como para a abordagem CSC. 

A ligação com o modelo do transporte restringido foi feita notando que para moléculas longas e flexíveis, cujo tamanho é superior ao do poro, o fator de impedimento difusivo, dado pelo rácio entre o coeficiente de difusão das moléculas no interior do poro e o coeficiente de difusão numa situação não confinada, tende para zero. Por outro lado, o fator de impedimento convectivo deverá ser igual a 1, assumindo que estas moléculas uma vez no interior de poros com menores dimensões deverão mover-se com a velocidade média do solvente. Verifica-se, pois, que o coeficiente de permeação intrínseca se torna igual ao coeficiente de partição.

O desenvolvimento teórico proposto neste trabalho teve por base as equações de Maxwell-Stefan. Com esta abordagem, verifica-se que as principais equações usadas nos modelos podem ser deduzidas com base no mesmo princípio teórico. A polarização de concentração pode ser obtida com recurso ao modelo do filme. Nestas condições, e considerando soluções diluídas e ideais, foi possível deduzir e aplicar a equação de Nernst-Planck na camada de polarização de concentração. Esta equação revela-se fundamental para determinar a polarização de concentração do DNA plasmídico dado que este tipo de moléculas apresenta uma elevada carga elétrica. A equação de Nernst-Planck introduz importantes efeitos na permeação de moléculas com carga. Em especial, os vários compostos iónicos em solução exercem uma influência mútua entre si, influência esta que se manifesta pelo surgimento de um potencial elétrico na camada de polarização. Assim, para além do fluxo de filtração e dos coeficientes de difusão, a concentração e valência elétrica dos iões desempenham um papel importante no estabelecimento dos gradientes de concentração na camada de polarização e consequentemente nos valores da permeação observada dos vários componentes da solução a filtrar.

No decorrer do trabalho verificou-se que as moléculas de RNA também exibem permeação em poros com dimensões inferiores às dimensões moleculares. Este facto leva assim à necessidade de considerar igualmente efeitos de deformação molecular para explicar a sua permeação. Neste domínio, o facto do desenvolvimento do modelo para o pDNA ter sido efetuado de um modo mais abrangente possibilitou a sua aplicação ao caso do RNA, uma molécula que revela igualmente um enorme potencial como fármaco de segunda geração, e onde os processos de membranas deverão ter aplicabilidade na sua produção. O modelo aplicado ao RNA permitiu obter previsões em concordância com os resultados experimentais. A descrição da sua polarização de concentração revelou também a importância da utilização da equação de Nernst-Planck, uma vez que a permeação de RNA na presença de pDNA revelou ser quase independente do fluxo de filtração, o que só pode ser explicado por efeitos de carga. Por outro lado, o modelo desenvolvido prevê, para as concentrações das espécies estudadas neste trabalho, que a influência da presença de RNA na permeação de pDNA é muito menos significativa, facto que também foi verificado na prática.

É importante referir que a dedução da equação de Nernst-Planck mostra o elevado número de simplificações que é necessário efetuar. Em especial, têm que ser consideradas soluções diluídas e ideais para a equação ser aplicável. Por outro lado, o efeito da pressão no potencial químico tem que ser desprezado, o que para solutos com elevados volumes parciais molares pode não ser o melhor procedimento. Em processos de ultrafiltração, em especial em operações de concentração, a aproximação de soluções diluídas e ideais pode igualmente apresentar-se como uma descrição pouco realista. No entanto, não é menos verdade que a complexidade inerente a processos de micro e ultrafiltração, em especial a ocorrência de colmatação e subsequente alteração físico-química da membrana, não permite garantir que mesmo um modelo teórico mais preciso possa descrever com exatidão todo o fenómeno de permeação. A introdução de novos termos na equação de Nernst-Planck aumenta consideravelmente a dificuldade de integração dos sistemas de equações diferenciais. Esta dificuldade surge pelo facto dos valores que se procuram calcular (as concentrações dos vários componentes no permeado) precisarem de ser conhecidos para proceder à integração das equações de forma direta. Para superar esta limitação foi desenvolvido um algoritmo que permite, de forma iterativa, estimar estes valores e ir efetuando novas estimativas mais precisas até se obter uma aceitável convergência. Naturalmente, à medida que um maior número de componentes é considerado nos cálculos a complexidade inerente é aumentada significativamente, pelo que o desenvolvimento de algoritmos mais eficientes é assim premente. 

No presente trabalho foi determinado o coeficiente de partição de moléculas longas e flexíveis em dois passos: pela representação da estrutura das moléculas por modelos FJC e CSC e pela simulação estocástica necessária para determinar a probabilidade de entrada destas estruturas em poros. Foi considerada como condição necessária de entrada no poro que a massa pontual mais perto do poro esteja projetada no seu interior, sendo a restante molécula forçada a permear pelo efeito de deformação causado pelas tensões de corte geradas pelo fluxo convectivo do solvente. A utilização deste procedimento permitiu obter previsões com uma satisfatória aproximação aos resultados experimentais. No entanto, verificou-se uma certa tendência para subestimar ligeiramente as permeações. Neste domínio, poderão obviamente ser efetuados alguns melhoramentos, como por exemplo na determinação experimental de condições de entrada em poros que possam descrever de forma mais realista o fenómeno de permeação. As representações FJC e CSC são igualmente representações que poderão ser demasiado simples para descrever a estrutura de pDNA e RNA. Por exemplo, nestas representações não é tido em conta o efeito do volume excluído, que pode desempenhar um papel importante. Estes dois efeitos poderão contribuir para que se verifiquem previsões de permeação geralmente mais elevadas do que as que se observam na prática. No entanto, do mesmo modo poderá ser referido que o incremento da complexidade na análise do problema gera incontornavelmente um enorme aumento no esforço, quer computacional quer de realização de ensaios experimentais, sem que exista uma clara certeza que a exatidão das previsões seja significativamente melhorada, o que se deve, tal como referido anteriormente, à elevada complexidade inerente aos processos de micro e ultrafiltração.

Os modelos teóricos desenvolvidos para estimar a permeação de pDNA consideram apenas a sua isoforma super-enrolada. Este facto deve-se em parte à escassez de dados na literatura que permitam estimar de forma precisa as propriedades das várias isoformas. Como trabalho futuro seria importante procurar alargar o modelo para permitir a sua aplicação diferenciada às várias isoformas de pDNA, que inevitavelmente são parte integrante dos lisados. Variações nos valores de permeação para diferentes isoformas, em processos de ultrafiltração, foi já verificada na prática \cite{zydneyiso} e a obtenção de um modelo teórico com capacidade de previsão constituiria uma ferramenta de inegável valor neste domínio, em especial tendo em consideração a importância da separação de isoformas num processo de produção de pDNA.

De um ponto de vista de aplicação prática foi possível obter uma elevada seletividade na sepração pDNA/RNA por ultrafiltração. Neste aspeto, os modelos desenvolvidos revelaram-se essenciais para estimar as condições operatórias necessárias. O modelo previu a utilização de membranas com poros de razoáveis dimensões (20--25\,nm) operadas a fluxos de filtração reduzidos. A utilização de baixos fluxos gera a possibilidade de reter o pDNA permitindo uma elevada permeação das moléculas de RNA. Por outro lado, devido aos reduzidos valores do coeficiente de difusão das moléculas de pDNA, fluxos baixos permitem obter uma menor polarização de concentração e consequente inferior grau de colmatação. Este facto pode gerar um aumento no tempo dos ciclos de utilização entre lavagens e uma redução na periodicidade de substituição de membranas. Por outro lado, minimizam a adsorção de compostos, facto que conduz a indesejadas alterações físico-químicas das membranas, o que pode levar a uma redução da reprodutibilidade entre filtrações. Este modo de operação a fluxo constante e de reduzido valor não é o mais utilizado, por exemplo em esquemas de purificação de proteínas. Para a sua implementação poderá ser necessário desenhar novos módulos de filtração que permitam efetuar um controlo preciso do fluxo de filtração. Por outro lado, fluxos reduzidos implicam a necessidade de utilizar uma maior área de membrana para obter uma produtividade semelhante. Assim, a viabilidade do processo terá que ser necessariamente avaliada tendo em conta as inerentes vantagens e desvantagens do método. No entanto, o processo de ultrafiltração/diafiltração desenvolvido no presente trabalho permite a eliminação da utilização de elevadas quantidades de agentes precipitantes, o que naturalmente tem o potencial de gerar um menor impacto ambiental e económico.   

No âmbito da produção e purificação de DNA plasmídico é essencial obter métodos de quantificação que permitam, de forma célere e eficiente, quantificar não só o plasmídeo como também os vários contaminantes ao longo do processo. Nesta matéria, o método cromatográfico de HIC, desenvolvido por Diogo \et\ \cite{diogo}, revela possuir ótimas características para tal, nomeadamente uma boa reprodutibilidade, robustez, ausência de pré-tratamento das amostras (com exceção da remoção do conteúdo sólido) e rapidez de execução. No presente trabalho de doutoramento sugere-se que o RNA possa ser igualmente analisado por este método, permitindo uma análise simultânea conjuntamente com o pDNA. Para controlo de qualidade da preparação final de pDNA deverá ser preciso efetuar uma análise mais precisa do conteúdo de RNA, no entanto, para controlar os rendimentos de remoção desta biomolécula em fases intermédias do processo de produção verifica-se que o método de HIC pode constituir uma excelente alternativa.

Na presente tese de doutoramento procurou-se estudar a remoção de sólidos, originados pelo processo de lise alcalina, recorrendo ao uso de uma operação de microfiltração. Outras processos alternativos podem ser considerados, como por exemplo centrifugação, filtração convencional e flutuação forçada. No entanto, a centrifugação é geralmente considerada como pouco viável à escala industrial. Filtração convencional e operações de flutuação forçada não permitem garantir a total remoção de conteúdo sólido, pelo menos de uma forma tão segura como a que se obtém com uma operação de microfiltração. Este fator pode ter importância em especial para situações em que a operação subsequente apresente uma reduzida tolerância à presença de sólidos na corrente de entrada, como por exemplo operações de cromatografia. No presente trabalho foi estudada a microfiltração de um lisado alcalino. Foram obtidos elevados rendimentos de recuperação de pDNA e completa remoção de sólidos aquando da utilização de uma membrana hidrofílica de Nylon com um poro nominal de 0.2\,\micro m, a um fluxo de filtração constante de 73\,L/h$\cdot$m$^2$. Durante a filtração dos lisados foi, contudo, observada alguma re-dissolução de gDNA durante o processo. Verificou-se que este efeito não poderia ser explicado apenas pela redução da força iónica do lisado durante o processo, e que teria de ser, portanto, resultante da degradação mecânica do conteúdo precipitado pela ação das tensões de corte geradas. De facto, observou-se que o gDNA re-dissolvido apresentou valores de permeação significativos numa membrana de ultrafiltração com um poro de tamanho reduzido ($\raioporo = 4$\,nm), confirmando a ideia de que a ação mecânica imposta no material precipitado leva a re-dissolução de pequenos fragmentos de gDNA, que como é sabido podem causar complicações em fases mais avançadas do processo. Este resultado está assim de acordo com a ideia de ser necessário impor, durante operações de separação sólido-líquido a seguir à lise, reduzidas tensões de corte para evitar a re-dissolução de contaminantes. Apesar disso, também se verificou que na segunda operação de membranas, a ultrafiltração, é possível remover de forma eficientemente estes fragmentos contaminantes.    

Nos ensaios de microfiltração, referidos no parágrafo anterior, observou-se a ocorrência de alguma colmatação das membranas. Devido à elevada quantidade de sólidos e de contaminantes dissolvidos nos lisados este fenómeno será sempre, em maior ou menor grau, uma inevitabilidade. Uma forma de procurar rentabilizar a operação e de reduzir o impacto ambiental e económico subjacente passa por desenvolver novas membranas que possam ser produzidas de forma económica e que sejam biodegradáveis. Nos nosso laboratórios estão a ser dados passos nesta direção como comprova um estudo recentemente publicado \cite{meues}.  
