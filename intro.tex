%!TEX root=tese.tex
\chapter{Introdução}
\label{chap:intro}

\section{Enquadramento do trabalho}
O trabalho descrito na presente tese de doutoramento aborda a integração de processos de membranas, nomeadamente a micro e ultrafiltração (MF e UF), em esquemas de purificação de DNA plasmídico (pDNA). Nos últimos anos, tem-se assistido a um enorme esforço de investigação nesta temática, o que se pode comprovar pelo elevado número de publicações cientificas e de patentes registadas \cite{carnes}. A principal motivação surge da possibilidade de utilizar estas biomoléculas em aplicações de terapia génica e na formulação de vacinas de DNA. Em termos simples, a terapia génica consiste no tratamento de uma doença pela introdução de nova informação genética nas células do paciente para compensar a deficiência ou ausência de determinados genes, ou para possibilitar uma nova função \cite{emery}. A evolução da terapia génica alcançou no ano transato um importante marco histórico, nomeadamente, a autorização da  EMA (``European Medicines Agency'') para a comercialização de um fármaco de terapia génica (Glybera), para o tratamento da deficiência da lipoproteína lipase \cite{wirth}. Existem inúmeras referências na literatura que se focam neste promissor, e simultaneamente controverso, tema \cite{mountain,emery,wirth,schleef}, assim como na formulação de vacinas de DNA \cite{liu03,liu11,kutzler}.\index{terapia génica}\index{vacinas DNA}\index{Glybera}

Os plasmídeos, juntamente com retrovírus e adenovírus, constituem os três tipos de vetores mais usados atualmente em ensaios clínicos de terapia génica (tabela~\ref{tab:vetores}). Se por um lado, a utilização de vetores virais permite obter eficiências de transfecção mais elevadas, existem algumas preocupações na comunidade científica quanto às questões da segurança do paciente durante um tratamento com este tipo de vetor \cite{wirth,duarte,mountain}. Neste respeito, o DNA plasmídico apresenta uma alternativa mais segura. Este facto terá contribuído para o crescente aumento na utilização de pDNA em ensaios clínicos, como se pode comprovar pelos valores apresentados na tabela~\ref{tab:vetores}. Apesar de nos últimos dois anos ter ocorrido um decréscimo de utilização dos três principais vetores, o pDNA foi o que registou um decréscimo menor. Por outro lado, é importante notar o aumento significativo na sua utilização comparando com os valores que se verificavam no ano 2000 \cite{mountain}. 
\index{vectores virais}\index{transfecção}
\begin{table}
\centering
	\caption[Percentagem de utilização de vetores em terapia génica.]{Percentagem de utilização, em ensaios clínicos de terapia génica, dos três vetores mais usados.}
	\label{tab:vetores}
\begin{threeparttable}
\begin{tabular*}{12cm}{l@{\extracolsep{\fill}} d{7} d{6} d{4} }
\toprule
Ano & \mc{1}{r}{Adenovírus} & \mc{1}{l}{Retrovírus} & \mc{1}{l}{pDNA} \\
\midrule
2000\tnote{a} & 19,3 & 53,0 & 5,3 \\
2007\tnote{b} & 24,9 & 21,7 & 18,3 \\
2011\tnote{c} & 24,9 & 20,7 & 18,7 \\
2013\tnote{d} & 23,5 & 19,1 & 17,7 \\
\bottomrule 
\end{tabular*}
\begin{tablenotes}
\item[a] Valores obtidos em \cite{mountain} 
\item[b] Valores obtidos em \cite{cai}
\item[c] Valores obtidos em \cite{meu3}
\item[d] Valores obtidos em \url{http://www.abedia.com/wiley/vectors.php}
\end{tablenotes}
\end{threeparttable}
\end{table}
%

A menor eficiência de transfecção conseguida com estas biomoléculas leva à necessidade de administrar elevadas doses deste vetor durante um único tratamento \cite{prather,gomes}. Por este motivo, surge a necessidade de desenhar processos de produção e purificação de DNA plasmídico que apresentem elevados rendimentos. Por outro lado, uma preparação final de DNA plasmídico deve obedecer de forma rigorosa às normas de segurança vigentes, que estabelecem limites mínimos, por regra muito reduzidos, da presença de contaminantes, tal como legislado pelas agências reguladoras do setor, como por exemplo a FDA (``U.S. Food and Drug Administration'') ou a EMA.\index{FDA}\index{EMA} 
Este facto leva à necessidade de desenhar esquemas de purificação de pDNA que devem estar essencialmente livres de compostos químicos perigosos, tais como cloreto de césio ou brometo de etídio, de solventes tais como isopropanol, fenol ou clorofórmio e de enzimas (por exemplo RNase A, proteinase K ou lisozima), frequentemente usados em protocolos laboratoriais \cite{prather,sousahplc,smrekar,duvaltff,sousabab,kahn,lander}. 
\index{enzima}%
Como será discutido, os processos de membranas apresentam um enorme potencial de utilização em diferentes fases de um processo de produção e purificação de DNA plasmídico, podendo suprir algumas das limitações referidas durante o desenvolvimento e implementação destes processos. Assim, no presente trabalho de doutoramento procura-se estudar a aplicação de duas operações de membranas (micro e ultrafiltração) em processos de produção de pDNA. O trabalho foi desenvolvido quer de um ponto de vista teórico, quer de um ponto de vista de aplicação prática. Na secção seguinte é feita uma descrição da possível integração destes processos num esquema de purificação de DNA plasmídico.

\section{Integração de processos de MF e UF na purificação de pDNA}
Vários artigos científicos têm vindo a ser publicados, onde os autores referem e discutem as principais etapas de um processo de produção de DNA plasmídico, desde a fase inicial do desenho dos plasmídeos e escolha de hospedeiro, até às fases finais de purificação e formulação do produto final. Destes estudos podem-se destacar as extensas revisões de Prather \et\ \cite{prather}, de Prazeres e Ferreira \cite{flowsheets} e de Carnes e Williams \cite{carnes}. Na figura~\ref{fig:flowsheets} encontra-se um esquema do processo de purificação de DNA plasmídico, adaptado das três referências indicadas. Os autores dividem o processo de purificação de pDNA, a seguir à operação de fermentação, em três etapas\footnote{É comum designar-se o conjunto de operações que se seguem à etapa fermentativa como o processo ``downstream'', sendo as operações que a precedem englobadas na fase de ``upstream''.}: isolamento primário, isolamento intermediário e purificação final. Como se pode constatar na figura~\ref{fig:flowsheets}, as operações de MF e UF podem ser usadas em praticamente todas as fases do processo.\index{downstream@``downstream''}\index{isolamento primário}\index{purificação final}
\index{Upstream@``Upstream''}\index{cromatografia!permita aniónica}\index{cromatografia!fase reversa}\index{cromatografia!exclusão molecular|see{SEC}}\index{cromatografia!leito expandido}\index{cromatografia!IMAC}  
\begin{figure}[!t]
\centering
\begin{tikzpicture}

%\draw[help lines] (0cm,0cm) grid (10cm,12cm);

% Define block styles  
\tikzstyle{materia}=[draw, fill=gray!20, text width=6.0em, text centered,
  minimum height=1.5em,drop shadow]
\tikzstyle{practica} = [materia, text width=16em, minimum width=10em,
  minimum height=3em, rounded corners, drop shadow]
\tikzstyle{texto} = [below right]
\tikzstyle{linepart} = [draw, thick, color=black!50, -latex', dashed]
\tikzstyle{line} = [draw, thick, color=black!50, -latex']
\tikzstyle{ur}=[draw, text centered, minimum height=0.01em]
 
% Define distances for bordering
\newcommand{\blockdist}{1.3}
\newcommand{\edgedist}{1.5}

\newcommand{\practica}[2]{node (p#1) [practica]
  {\scriptsize{Colheita de células} \\{\scriptsize\textit{#2}}}}

% Draw background
\newcommand{\background}[5]{%
  \begin{pgfonlayer}{background}
    % Left-top corner of the background rectangle
    \path (#1.west |- #2.north)+(-0.5,0.5) node (a1) {};
    % Right-bottom corner of the background rectanle
    \path (#3.east |- #4.south)+(+0.5,-0.25) node (a2) {};
    % Draw the background
    \path[fill=gray!7.5,rounded corners, draw=black!50, dashed]
      (a1) rectangle (a2);
    \path (a1) node (u1)[texto]
      {\scriptsize\textit{#5}};
  \end{pgfonlayer}}

\newcommand{\transreceptor}[3]{%
  \path [linepart] (#1.east) -- node [above]
    {\scriptsize Transreceptor #2} (#3);}


  % Draw diagram elements
\path node[anchor=north] at (5,12) (p1) [practica] {%
Colheita de células\\%
\scriptsize{
\begin{flushleft}
$\ast$ Centrifugação\\%
$\ast$ \textbf{MF} \\%
\end{flushleft}
}};

\path (p1.south)+(0.0,-0.5) node[anchor=north] (p2) [practica] {%
Lise celular\\%
\scriptsize{
\begin{flushleft}
$\ast$ Química (alcalina/detergente)\\%
$\ast$ Térmica\\%
$\ast$ Mecânica (não recomendado)\\
\end{flushleft}
}};

\path (p2.south)+(0.0,-0.5) node[anchor=north] (p3) [practica] {%
Separação sólido-líquido\\%
\scriptsize{
\begin{flushleft}
$\ast$ Filtração (\textbf{MF}, materiais inertes)\\%
$\ast$ Flutuação\\%
$\ast$ Centrifugação (não recomendado)\\%
\end{flushleft}
}};

\path (p3.south)+(0.0,-1) node[anchor=north] (p4) [practica] {%
Concentração/Troca de tampão/Pré-purificação\\%
\scriptsize{%
\begin{flushleft}
$\ast$ Precipitação (sais, PEG, Álcoois, CTAB)\\%
$\ast$ \textbf{UF}\\%
$\ast$ Extração líquido-líquido\\%
$\ast$ Cromatografia de leito expandido\\
\end{flushleft}
}};

\path (p4.south)+(0.0,-1) node[anchor=north] (p5) [practica] {%
Purificação/Separação isoformas\\%
\scriptsize{%
\begin{flushleft}
$\ast$ Cromatografia\\%
\ \ \ \ $\circ$ AE, RP, HIC, SEC, IMAC, Afinidade\\
\ \ \ \ $\circ$ Cromatografia de membrana\\
$\ast$ Adsorção seletiva de impurezas\\
$\ast$ \textbf{UF}\\%
\end{flushleft}
}};

\path (p5.south)+(0.0,-0.5) node[anchor=north] (p6) [practica] {%
Concentração/Esterilização/Troca de tampão\\%
\scriptsize{%
\begin{flushleft}
$\ast$ \textbf{MF}\\
$\ast$ \textbf{UF}\\
$\ast$ Precipitação\\
$\ast$ SEC\\
\end{flushleft}
}};

\background{p1}{p1}{p3}{p3}{Isolamento primário}
\background{p4}{p4}{p4}{p4}{Isolamento intermediário}
\background{p5}{p5}{p6}{p6}{Purificação final}

\end{tikzpicture}
\caption[Esquema genérico de um processo de purificação de DNA plasmídico.]{Esquema genérico de um processo de purificação de DNA plasmídico, adaptado de \cite{prather,flowsheets,carnes}. Operações de MF e UF podem ser encontrados em praticamente todas as fases do processo. MF - Microfiltração, UF - Ultrafiltração, AE - Cromatografia de permuta aniónica, RP - Cromatografia de fase reversa, HIC - Cromatografia de interação hidrofóbica, SEC - Cromatografia de exclusão molecular, IMAC - Cromatografia de afinidade com metal imobilizado.}
\label{fig:flowsheets}
\end{figure}   

Na grande maioria dos processos de produção de pDNA reportados na literatura, o plasmídeo de interesse é produzido por via fermentativa. A primeira etapa crítica do processo de ``downstream'' é a lise celular. Esta operação é necessária para abrir as paredes das células e assim libertar o plasmídeo em solução. As principais técnicas usadas são a denominada lise térmica \cite{holmes,zhu,mahony,lander} e a lise alcalina \cite{birnboim,chamsart,urthaler,meacle}. O método de lise mais usado continua a ser o método de lise alcalina, originalmente proposto por Birnboim e Doly \cite{birnboim}, sendo igualmente o método escolhido no âmbito do presente trabalho. Após aplicação deste método obtém-se uma elevada quantidade de precipitados, cerca de 100\,g/L num procedimento  padrão \cite{theo}, e uma elevada quantidade de contaminantes dissolvidos, sendo o RNA o contaminante presente em maior quantidade. A necessidade de remover o conteúdo sólido formado proporciona a possibilidade de aplicação de operações de microfiltração. Outras técnicas alternativas podem ser consideradas, por exemplo operações de centrifugação \cite{flowsheets}, de filtração com materiais inertes \cite{urthaler} ou por métodos de flutuação, como recentemente reportado \cite{blom}.\index{lise térmica}\index{lise alcalina!referências bibliográficas}\index{lise celular|see{lise alcalina, lise térmica}} 

Numa segunda fase, após remover o conteúdo sólido, é geralmente necessário proceder a uma pré-purificação do plasmídeo, uma fase a que se dá o nome de isolamento intermediário. Nesta etapa são efetuadas operações que visam preparar a corrente do processo para as operações de purificação final. Por exemplo, em regra geral é necessário concentrar o pDNA, trocar de tampão e remover alguns contaminantes. A ultrafiltração, devido às suas características, pode ser usada nesta fase do processo. Outras técnicas podem ser, no entanto, consideradas como por exemplo operações de precipitação seletiva, de extração líquido-líquido e de cromatografia de leito expandido \cite{flowsheets}. \index{isolamento intermediário}\index{troca de tampão}  

Podem ser ainda enumeradas outras aplicações de processos de MF e UF em diferentes fases do processo, como por exemplo na colheita de células após a fermentação, na esterilização do produto final e até mesmo na fase de purificação final \cite{zydneyiso}. No âmbito do presente trabalho, as operações estudadas dizem respeito à aplicação da microfiltração para a remoção de sólidos formados no decorrer da lise alcalina e à aplicação de ultrafiltração para concentrar, trocar de tampão e pré-purificar o plasmídeo antes da fase final de purificação.\index{microfiltração!operação estudada}\index{ultrafiltração!operação estudada}
Estas duas operações são igualmente abordadas de um ponto de vista teórico. Os resultados obtidos podem também ser úteis no desenvolvimento e otimização de outras operações de MF e UF em diferentes fases do processo. Na secção seguinte é feita uma revisão bibliográfica dos avanços que têm vindo a ser feitos no contexto da aplicação de operações de MF e UF em processos de produção de pDNA (secção~\ref{sub:processosrevisão}), assim como os avanços alcançados na modelação da permeação destas biomoléculas em membranas (secção~\ref{sec:modulaçãopermeação}).

\section{Revisão bibliográfica}
\label{sec:membranasnaprodução}

\subsection{MF e UF em processos de produção de pDNA}
\label{sub:processosrevisão}
Os processos de separação baseados em tecnologias de membranas são dos processos mais ubíquos em biotecnologia \cite{rathore}. Este facto deve-se essencialmente à enorme quantidade de operações que podem ser efetuadas segundo esta tecnologia, onde se podem incluir operações de separação sólido-líquido, concentração, troca de tampão, purificação e esterilização. Outras qualidades que tornam estes processos atrativos podem ser enumeradas, como por exemplo o facto destas operações apresentarem custos, quer de investimento inicial quer de funcionamento, regra geral mais reduzidos quando comparados com os custos associados a outros processos alternativos \cite{rathore}. Pela sua natureza de funcionamento, são igualmente processos com um impacto ambiental reduzido \cite{freitas}. Pode ainda ser referido que os processos de separação com membranas apresentam, em determinadas aplicações, características que possibilitam um scale-up linear \cite{reis}, assim como a possibilidade de automatização e funcionamento em modo contínuo, o que de um ponto de vista de produção em larga escala é sempre desejável \cite{plumb}.\index{tecnologias de membranas!características} 

Por esse motivo tem vindo a ser estudada a aplicação de tecnologias de membranas em várias etapas de um processo de produção de DNA plasmídico. Num dos primeiros estudos publicados nesta matéria \cite{theo}, Theodossiou \et\ avaliaram, em 1997, o desempenho de uma operação de filtração na clarificação de lisados obtidos pelo método da lise alcalina, em que uma grande quantidade de sólidos é formada.\index{microfiltração!revisão bibliográfica} 
Tal como referido pelos autores do estudo, enquanto que por norma, em laboratório, estes sólidos podem ser facilmente removidos com recurso a operações de centrifugação descontínua, na produção à escala industrial a centrifugação de lisados provenientes da lise alcalina levanta sérias considerações de impraticabilidade, o que se deve em parte às elevadas tensões de corte geradas no interior de equipamentos industriais de centrifugação. Este facto foi posteriormente referido por uma série de outros autores \cite{blom,mahony,kong10,duval2,zhu}.\index{centrifugação!impraticabilidade} 
No estudo, foi utilizado um largo conjunto de filtros, de vários materiais e tamanhos de poro, tendo-se verificado a existência de algumas complicações. Por exemplo, devido à natureza deformável dos precipitados formados, a utilização de filtros com poros superiores a 5\,\micro m parece ser desaconselhável, em parte porque não se garante a total retenção dos sólidos presentes, e as tensões de corte geradas durante a intrusão destes sólidos nos poros provoca a re-dissolução de alguns contaminantes, de onde se destaca o DNA genómico. Por outro lado, mesmo com o filtro com poros de menores dimensões (no caso do referido estudo 5\,\micro m) os autores verificaram uma incompleta remoção do material precipitado. Para assegurar uma total remoção de sólidos, os autores recorreram ao uso de terra de diatomáceas. No entanto esta alternativa provoca duas novas desvantagens, nomeadamente o decréscimo do rendimento da operação, provocada em parte pelo aumento da adsorção de moléculas de pDNA, e o aumento de resíduos sólidos formados.  

Um pouco mais tarde, no ano 2000, surgiu provavelmente o primeiro estudo sobre a aplicação de uma operação de ultrafiltração na purificação de DNA plasmídico, sendo o trabalho da autoria de Kahn \et\ \cite{kahn}. Neste trabalho, Kahn \et\ \cite{kahn} conseguiram obter pDNA bastante puro mas somente após submeterem os lisados a extensos períodos de incubação, para assim obterem uma degradação significativa do RNA presente. Neste estudo ficou pela primeira vez evidenciada a dificuldade da separação entre pDNA e RNA com recurso a uma operação de ultrafiltração, e apesar dos extensos períodos de incubação do lisado levarem a uma elevada degradação do RNA, não é claro que o mesmo não venha a acontecer, em parte, às moléculas de pDNA. Por outro lado, extensos períodos de incubação, num ponto de vista de aplicação industrial, podem gerar alguns problemas na automatização e controlo do processo.\index{ultrafiltração!revisão bibliográfica|(}  

Ainda no mesmo ano, Levy \et\ \cite{levy00} publicaram um estudo onde  membranas de microfiltração de nitrocelulose foram utilizadas para adsorver seletivamente gDNA, reduzindo em algumas experiências o conteúdo deste contaminante de mais de 27\% para menos de 1\%. Para forças iónicas iguais ou superiores a 1.5\,M de NaCl os autores verificaram ainda alguma redução do conteúdo de RNA. A mesma ideia foi explorada por Kendall \et\ \cite{kendall} um pouco mais tarde. Nestes estudos existe assim uma tentativa de efetuar duas operações num único passo, nomeadamente obter de forma simultânea uma separação sólido-líquido e uma remoção de contaminantes, sendo esta uma importante vantagem do método. No entanto, deve ser notado que as membranas apresentam por regra uma baixa capacidade para atuar como adsorventes. Por outro lado, a adsorção de contaminantes nas membranas pode levar à alteração físico-química das mesmas, o que pode conduzir a uma colmatação prematura e uma redução da reprodutibilidade entre filtrações.\index{adsorção!nitrocelulose}\index{nitrocelulose|see{adsorção}} 

Em 2003, Eon-Duval \et\ \cite{duvaltff} publicaram um estudo onde abordaram uma operação de ultrafiltração com o intuito de remover a grande maioria dos contaminantes presentes após a lise alcalina, com especial enfoque para o RNA. Novamente, os autores observaram uma reduzida capacidade do processo para separar eficientemente pDNA e RNA. Para superar esta limitação os autores acoplaram à operação de ultrafiltração uma operação de precipitação seletiva com \tce{CaCl2}. Tal como verificado no estudo de Kahn \et\ \cite{kahn}, após prévia remoção de RNA é possível obter uma elevada remoção de contaminantes com uma única operação de ultrafiltração. O método desenvolvido, apesar de eficiente, tem a clara desvantagem da necessidade de utilização de grandes quantidades de \tce{CaCl2}, o que aumenta significativamente o custo e impacto ambiental do processo.\index{cloreto de cálcio!precipitação}

Em 2004, Kepka \et\ \cite{kepka} utilizaram com sucesso uma operação de ultrafiltração para trocar de tampão e concentrar um lisado, proveniente da lise alcalina, antes de o submeterem a um processo de extração líquido-líquido. A boa capacidade de processos de ultrafiltração para efetuar operações de concentração e de troca de tampão/dessalinização foi assim demonstrada. A redução de volume e da força iónica do lisado conseguida no passo de ultrafiltração revelou-se essencial para a exequibilidade do processo proposto pelos autores.

Em 2006 surgiu talvez a primeira publicação referente a uma operação de esterilização com recurso à microfiltração, no âmbito da produção de DNA plasmídico. O trabalho foi desenvolvido por Kong \et\ \cite{kong06}, onde a filtração de soluções puras de pDNA foi efetuada com membranas de PVDF e PES com 0.22\,\micro m e 0.2\,\micro m de poro, respetivamente. Um novo trabalho sobre esta mesma temática viria a ser publicado mais tarde por parte dos autores \cite{kong10}. Os referidos estudos vieram comprovar a possibilidade de utilizar uma operação de microfiltração para proceder ao passo final de esterilização da preparação de pDNA. Os autores identificam potenciais problemas nesta operação, em especial à medida que o tamanho das moléculas de pDNA aumenta, o que pode levar a uma excessiva retenção e consequente perda de rendimento. Verificou-se igualmente uma dependência quase linear entre a permeação de pDNA e o tamanho das moléculas estudadas, bem como um aumento de permeação com o aumento da força iónica do meio.\index{esterilização}

Em 2008 \cite{freitasposter} e em 2009 \cite{freitas}, Freitas \et\ publicaram dois estudos sobre a aplicação de uma operação de ultrafiltração, com diafiltração, no isolamento intermediário de DNA plasmídico, procurando esta operação constituir uma alternativa viável a outras operações, nomeadamente operações de precipitação e de extração líquido-líquido. Apesar dos bons rendimentos de recuperação de pDNA e de remoção de proteínas e endotoxinas obtidos, os autores verificaram novamente baixa remoção de RNA no passo de ultrafiltração. Este resultado, em concordância com estudos anteriores \cite{kahn,duvaltff} apresenta-se assim como principal limitação do método, uma vez que em termos de impacto ambiental e económico, o processo de ultrafiltração permitiu obter resultados mais satisfatórios, quando comparado com os processos alternativos estudados.   

O trabalho desenvolvido pelos vários autores, sobre a aplicação de tecnologias de separação com membranas num processo de produção de DNA plasmídico, veio comprovar que estes processos podem constituir alternativas atrativas quer nos isolamentos primário e intermediário, quer na fase final de esterilização. No entanto, em 2011, Latulippe e Zydney \cite{zydneyiso} comprovaram também a possibilidade de se conseguir uma separação de isoformas de pDNA recorrendo à ultrafiltração, operação até aí apenas possível com outras tecnologias de maior resolução (tais como cromatografia, eletroforese em gel de agarose, entre outras). O processo desenvolvido baseia-se na diferença entre os valores de fluxo crítico para as várias isoformas de plasmídeo, sendo o fluxo crítico o valor de fluxo para o qual as moléculas apresentam um aumento súbito de permeação pela membrana, provocado pela deformação induzida pelas tensões de corte geradas à entrada do poro. 

Mais recentemente, Sun \et\ \cite{sun13} desenvolveram um processo de produção de DNA plasmídico em que utilizam várias operações de membranas: uma primeira filtração do lisado numa membrana com $1.0$\,\micro m de poro, uma operação de ultrafiltração para remover RNA de baixo peso molecular, concentrar e trocar de tampão, uma operação de esterilização com uma membrana de 0.2\,\micro m de poro e ainda uma operação de ultrafiltração para efetuar uma concentração final. Este trabalho indica com clareza a utilidade das tecnologias de membranas ao longo de um processo de produção de DNA plasmídico.\index{ultrafiltração!revisão bibliográfica|)}

Os processos de separação com membranas são parte integrante dos esquemas de purificação de proteínas \cite{reis}, e o trabalho que tem vindo a ser desenvolvido na última década e meia veio comprovar que para a produção de DNA plasmídico o cenário pode vir a ser o mesmo. Contudo, ainda existe um grande desconhecimento a nível teórico dos processos envolvidos na filtração de moléculas como o DNA plasmídico ou o RNA. Assim no presente trabalho de doutoramento, para além de terem sido investigados os principais aspetos relacionados com a aplicabilidade prática das técnicas de filtração com membranas, nomeadamente ao nível da separação sólido-líquido por microfiltração e purificação intermédia por ultrafiltração, deu-se particular relevância ao estabelecimento de modelos teóricos que permitam melhor compreender os processos de separação estudados. 

\subsection{Modelação da permeação de DNA plasmídico em membranas}
\label{sec:modulaçãopermeação}
O fenómeno da permeação de moléculas em poros de pequenas dimensões é uma matéria de interesse fundamental no estudo de operações de separação com membranas. Uma descrição matemática deste fenómeno é sempre desejável, em primeiro lugar porque um modelo matemático pode ser confrontado com resultados experimentais, podendo assim validar uma teoria ou ideia sobre o modo como as moléculas em estudo se comportam perante um cenário de permeação em poros. Em segundo lugar, a obtenção de um modelo matemático validado experimentalmente pode possibilitar prever, com um certo grau de exatidão, a permeação de solutos num processo de separação com membranas facilitando assim operações de otimização, aumento de escala, automatização e controlo. Alternativamente, a obtenção de previsões teóricas constitui um termo comparativo com os resultados experimentais facilitando assim uma interpretação dos mesmos, o que em última instância pode conduzir a uma melhor compreensão dos fenómenos envolvidos e posterior otimização da operação.  

Provavelmente um dos artigos mais citados, sobre o transporte restringido de solutos esféricos através de poros de pequenas dimensões, teoria que serviu de base para os modelos físicos desenvolvidos neste trabalho, seja o de Deen \cite{deen}, onde o autor faz uma revisão dos resultados obtidos neste âmbito.\index{transporte restringido!referências bibliográficas} Mais recentemente, Dechadilok e Deen \cite{dechadilok}, publicaram uma revisão mais atual sobre esta temática. Apesar dos primeiros avanços na teoria do transporte restringido, que datam do início da década de 50, e do posterior modelo de Deen, terem tido como motivação o estudo do movimento dos componentes do sangue em capilares \cite{moraotese}, os resultados obtidos acabaram por se revelar de grande interesse para a área das tecnologias de separação com membranas \cite{moraodes,moraompa,opong,rosa,combe,bowen96,bowen98,mohammad,pinho}. 

No entanto, os resultados que se obtêm por este modelo, posteriormente melhorado por vários autores \cite{dechadilok,moraodes,moraompa,bowen97,bowen02}, são apenas válidos para situações em que as dimensões do soluto não ultrapassam as dimensões do poro, e o soluto em si pode ser visto como uma esfera rígida (isto é, que não sofre deformação). Dentro desta categoria podem ser englobadas moléculas como proteínas, que possuem um núcleo hidrofóbico denso, originando uma estrutura globular com reduzida flexibilidade molecular \cite{latuiecr}. No caso das moléculas de DNA plasmídico, é sabido que estas apresentam valores de permeação significativos através de poros com dimensões bastante inferiores aos seus raios hidrodinâmicos. De facto, tal como mostrado recentemente por Arkhangelsky \et\ \cite{ark11}, sob suficiente pressão hidrostática observa-se a existência de permeação de plasmídeos em poros com dimensões cerca de 30 vezes inferiores. Naturalmente, tratar uma molécula de pDNA como uma esfera rígida é uma abordagem claramente insuficiente para a obtenção de modelos com boa capacidade de previsão.\index{esfera rígida} 

Pelo referido, surge a necessidade de abordar o problema de uma diferente forma. Os principais avanços nesta temática foram feitos por Zydney \et\ numa série de publicações \cite{latu07,latu09,latusalt,latuiecr,ager,zydneyiso}. \index{DNA plasmídico!revisão bibliográfica|(}
A primeira tentativa de modelação da permeação de DNA plasmídico em membranas de ultrafiltração surgiu em 2007 \cite{latu07}, com a publicação de um estudo em que os autores começaram por obter dados da permeação de um plasmídeo com 3.0\,kbp através de 3 membranas de ultrafiltração com diferentes ``cut-offs''. Verificou-se, nesse trabalho, a existência de um valor de fluxo de permeado a partir do qual o plasmídeo começava a permear pela membrana. Com base nesta observação os autores propuseram o conceito de fluxo crítico (\fluxocritico) para descrever a permeação de pDNA.\index{fluxo crítico|(} 
Este conceito indica que a partir de um determinado valor de fluxo de filtração (fluxo crítico) as tensões de corte geradas à entrada dos poros provocam a necessária deformação molecular para permitir a ocorrência de permeação. Os autores procuraram assim ajustar, aos valores de \fluxocritico\ determinados experimentalmente, o modelo desenvolvido por Daoudi e Brochard \cite{daoudi}:
\begin{equation}
\label{eq:daoudi}
\fluxocritico=\frac{\varepsilon \boltzman T}{\raioporo^{2}\viscosidade}
\end{equation}
onde \boltzman\ é a constante de Boltzmann, $T$ a temperatura absoluta, \viscosidade\ a viscosidade da solução, $\varepsilon$ a porosidade da membrana e \raioporo\ o raio do poro. No entanto, verificou-se que a previsão do fluxo crítico, dada pela equação~\ref{eq:daoudi}, sobrestimava em mais de 3 ordens de grandeza os resultados experimentais. Os autores propuseram assim uma modificação ao modelo de Daoudi e Brochard que proporcionou um melhor ajuste:
\begin{equation}
\label{eq:latu07}
\fluxocritico=\left(\frac{\pi^{2}D}{\raiogiracao^{2}}\right)\left(\frac{\varepsilon \raiosuperhelice^{3}}{\raioporo^{2}}\right)
\end{equation}
onde $D$ é o coeficiente de difusão, \raiogiracao\ é o raio de giração do plasmídeo e \raiosuperhelice\ é o seu raio da super-hélice. Com este novo modelo obteve-se um ajuste dentro da mesma ordem de grandeza. Os autores verificaram igualmente uma aparente independência das permeações com a velocidade de agitação. 

No ano seguinte, os mesmos autores publicaram um estudo sobre a permeação de um plasmídeo com 3.0\,kbp, através de membranas de ultrafiltração, em diferentes condições de força iónica \cite{latusalt}. O modelo representado na equação~\ref{eq:latu07} prevê variações no valor de \fluxocritico\ com variações na força iónica da solução através das variáveis \raiogiracao, \raiosuperhelice\ e $D$, sendo que as duas primeiras diminuem e a última aumenta com o aumento da força iónica \cite{latusalt}. Os autores verificaram uma enorme influência da concentração de \tce{NaCl} na permeação das moléculas de pDNA, tendo verificado uma permeação 80 vezes superior para um aumento da concentração de 1\,mM para 150\,mM de \tce{NaCl}. O modelo representado na equação~\ref{eq:latu07} apresentou bons resultados na previsão de \fluxocritico, mesmo tendo em conta o efeito do sal na permeação. 

Em 2009 \cite{latu09} o modelo proposto, dado pela equação~\ref{eq:latu07}, foi corrigido pelos mesmos autores. O novo modelo obtido tem a particularidade de prever uma independência do fluxo crítico em relação ao raio de giração das moléculas de plasmídeo, o que os autores identificaram como uma evidência experimental:
\begin{equation}
\label{eq:latu09}
\fluxocritico=\frac{\pi}{6}\left(\frac{\varepsilon}{\raioporo^{2}}\right)\left(\frac{\beta^{3}\deborahcritico}{\lambda}\right)\left(\frac{\boltzman T}{\viscosidade}\right)
\end{equation}
onde \deborahcritico\ é o número de Deborah crítico, $\lambda$ é o rácio entre o raio hidrodinâmico (\raiostokes) e o raio de giração (\raiogiracao) do plasmídeo e $\beta$ um parâmetro empírico. A ausência da forte dependência de \raiogiracao\ no valor de \fluxocritico\ está em melhor concordância com os resultados obtidos, em que a permeação de três plasmídeos de diferentes tamanhos se revelou estatisticamente semelhante. Ainda no mesmo ano \cite{latuiecr}, um novo estudo sobre a importância da flexibilidade de biopolímeros em processos de ultrafiltração viria a ser publicado pelos mesmos autores. Nesse estudo apresentaram-se resultados obtidos com pDNA e proteínas pegiladas.\index{fluxo crítico|)}

Ainda o mesmo grupo de investigação viria a publicar nesse ano um estudo sobre a influência da carga das membranas na permeação das moléculas de pDNA \cite{ager}, que como é bem sabido possuem uma elevada carga negativa. Os autores obtiveram valores de \fluxocritico\ muito superiores aquando da utilização de membranas carregadas negativamente, obtidas por um processo de modificação química, por comparação com os valores obtidos nessas mesmas membranas sem carga (não modificadas). Os autores atribuíram este fenómeno à ocorrência de interações eletrostáticas entre a membrana e o plasmídeo, ideia esta suportada pela atenuação desta diferença durante a filtração de soluções com maior força iónica (150\,mM \tce{NaCl}). O seguinte modelo foi sugerido para o cálculo da permeação observada (\permobs) do pDNA:
\begin{equation}
\label{eq:ager}
\ln{\left(\permobs\right)}=-\left( \frac{\Delta G_{\mathrm{config}}}{\boltzman T}\right)-\frac{Ze}{I_{0}\left(\kappa \raioporo\right)}\psi_{0}
\end{equation}
onde $\Delta G_{\mathrm{config}}$ representa a variação de energia livre associada com a alteração da conformação da molécula quando entra no poro, $Z$ é o seu número efetivo de cargas, $e$ é a carga de um eletrão, $I_{0}$ é a função de Bessel modificada de ordem zero, $\kappa$ é o inverso do comprimento de Debye e $\psi_{0}$ é o potencial zeta da membrana. O ajuste deste modelo aos pontos experimentais levou os autores a propor que a partição das moléculas de pDNA é determinada pela entrada apenas de uma pequena parte do plasmídeo no poro (um comprimento de persistência ou aproximadamente 50\,nm), com o resto da molécula a ser conduzida para o poro pelo fluxo convectivo do solvente.\index{DNA plasmídico!revisão bibliográfica|)}

A teoria desenvolvida nos artigos descritos em cima teve sempre como base a noção de que a flexibilidade inerente às moléculas de pDNA faz com que a partir de um certo fluxo de filtração os gradientes de velocidade que se formam junto aos poros da membrana são suficientes para provocar uma deformação nas moléculas, o que possibilita a sua permeação em poros com dimensões muito inferiores às dos raios hidrodinâmicos das moléculas. Foi esta ideia que esteve na origem do trabalho, publicado em 2011 por Latulippe e Zydney \cite{zydneyiso}, onde pela primeira vez se demonstrou a separação de isoformas de \mbox{pDNA} numa operação de ultrafiltração.\index{tecnologias de membranas!separação de isoformas}\index{ultrafiltração!separação de isoformas} 
De facto esta separação foi possível porque os autores verificaram experimentalmente que as diferentes isoformas apresentam diferentes valores de \fluxocritico, ou seja apresentam diferentes graus de flexibilidade. Neste estudo foi estabelecido um paralelismo com outros dois processos de separação alternativos: a eletroforese horizontal em gel de agarose, onde as isoformas apresentam diferentes mobilidades eletroforéticas, e a cromatografia de exclusão molecular, onde o que difere são os volumes de retenção. 

Apesar dos avanços feitos pelo grupo de investigação de Zydney \et\ terem fornecido um elevado conjunto de novos conceitos e uma nova visão sobre esta temática, pode ser mostrado experimentalmente que o conceito de fluxo crítico não é completamente verificado na prática, ou seja, é possível demonstrar que a permeação de pDNA não aumenta subitamente de valores de permeação reduzidos para a ocorrência de permeação total quando o fluxo de filtração ultrapassa o valor do fluxo crítico. A noção de fluxo crítico prevê que a permeação de pDNA só deve assumir valores ou muito reduzidos ou muito elevados. No entanto, na prática não é incomum obter permeações intermédias de pDNA em membranas de UF, sendo este um resultado que não é previsto pelo conceito de fluxo crítico. Por outro lado, não parece haver uma unificação dos modelos, ou seja, as várias equações não são dedutíveis a partir de um mesmo conceito teórico, e a sua formulação foi sendo feita com algumas noções semi-empíricas, para que as previsões se pudessem aproximar aos resultados experimentais. Este facto leva a que seja difícil estender a análise a outros tipos de moléculas de interesse, como por exemplo o RNA. Na presente tese de doutoramento é feita uma tentativa de superar estas limitações dos modelos de Zydney \et\ enunciados anteriormente. Para isso, procura-se estabelecer um paralelismo com um dos modelos mais citados e utilizados na área das operações de MF e UF, o denominado modelo do transporte restringido \cite{deen,dechadilok}, discutido no capítulo~\ref{chap:teov3}.\index{transporte restringido!referências bibliográficas} 
É igualmente intenção do presente trabalho procurar formular conceitos teóricos que possam permitir uma melhor adaptação dos modelos desenvolvidos a outros tipos de moléculas, sendo o RNA uma molécula de especial interesse, não só por ser um dos principais contaminantes num processo de produção de pDNA, como também pelo crescente interesse na sua utilização para fins terapêuticos.  

\section{Objetivos e estrutura da tese}
O objetivo principal da presente tese de doutoramento foi o de desenvolver o conhecimento na área da aplicação de operações de MF e UF em processos de produção de pDNA. Em especial procurou-se estudar a aplicação de uma operação de microfiltração para efetuar a remoção de sólidos formados na etapa de lise celular, e uma operação de ultrafiltração na fase do isolamento intermediário do DNA plasmídico. O presente trabalho teve também como objetivo desenvolver modelos teóricos com capacidade de prever a permeação de pDNA em membranas de micro e ultrafiltração. Procurou-se que o desenvolvimento teórico pudesse ser efetuado de uma forma abrangente para permitir a sua utilização em diferentes operações de membranas e para diferentes tipos de solutos. Em seguida é apresentado um pequeno resumo dos capítulos que integram a presente tese de doutoramento. 
\begin{description}
	\item[Capítulo 1] Neste presente capítulo é feito o enquadramento do trabalho explicitando as principais motivações e objetivos do mesmo. É feita uma revisão bibliográfica procurando identificar e discutir as principais publicações existentes na área de aplicação da presente tese, nomeadamente, no desenvolvimento de operações de MF e UF para integração em processos de produção de pDNA e modelação da permeação desta biomolécula em membranas. 
	\item[Capítulo 2] Abordam-se os fundamentos teóricos inerentes à realização do presente trabalho. Procura-se desenvolver o material exposto de forma sequencial e integrada. As principais equações e fundamentos usados são deduzidos, assim como as suas limitações indicadas e discutidas. Os principias tópicos abordam maioritariamente a descrição da transferência de massa em operações de MF e UF. A análise é estendida para o caso de moléculas longas e flexíveis.   
	\item[Capítulo 3] Discutem-se os aspetos práticos mais relevantes inerentes à realização do presente trabalho. Não se procura neste capítulo fazer uma extensa exposição de todos os métodos e procedimentos laboratoriais usados, mas sim expor os resultados práticos mais relevantes com aplicação transversal. Os detalhes práticos mais específicos referentes a um determinado capítulo são fornecidos nesse mesmo capítulo (capítulos~\ref{chap:art1}--\ref{chap:art4}). 
	\item[Capítulo 4] Desenvolve-se a modelação da permeação de moléculas longas e flexíveis em poros de pequenas dimensões. É estudada a permeação de duas macromoléculas, o plasmídeo \pUC\ na isoforma linear e o dextrano T2000. É elaborado um modelo de transferência de massa que pela primeira vez complementa os modelos de transporte restringido, possibilitando a sua aplicação para situações em que as dimensões dos solutos são superiores às dimensões dos poros. São utilizados métodos estocásticos para determinar o coeficiente de partição deste tipo de moléculas. O texto exposto foi publicado no \emph{Journal of Membrane Science} \cite{meu1}.
	\item[Capítulo 5] Estuda-se a ultrafiltração de DNA plasmídico na isoforma superenrolada, pDNA (sc). O modelo desenvolvido no capítulo~\ref{chap:art1} é usado para servir de base à construção de um novo modelo de transferência de massa. Este novo modelo permite contabilizar efeitos de carga na permeação de pDNA (sc) em membranas de ultrafiltração, o que tem especial interesse dada a elevada carga elétrica destas moléculas. É igualmente proposto um novo modelo de representação destas macromoléculas circulares (modelo CSC), que permite obter previsões mais exatas das suas dimensões. As previsões do modelo desenvolvido são comparadas com resultados experimentais de permeação de pDNA(sc). O fenómeno frequentemente observado da adsorção de plasmídeo em membranas, e também nos ensaios realizados, é interpretado em termos teóricos. O texto exposto foi publicado no \emph{Journal of Membrane Science} \cite{meu2}
	\item[Capítulo 6] Neste capítulo é desenvolvido um estudo de aplicação de membranas de  microfiltração para a remoção dos sólidos originados durante o processo de lise alcalina. É estudada igualmente uma operação de ultrafiltração subsequente para efetuar uma concentração do lisado filtrado. O modelo desenvolvido na capítulo~\ref{chap:art2} é usado para orientar a escolha das membranas e para interpretar os resultados experimentais. Neste capítulo, é modelada a permeação de diferentes tipos de moléculas de RNA, como cadeias flexíveis, obtendo-se assim uma ampliação do espectro de aplicação do modelo desenvolvido no capítulo~\ref{chap:art1}, para este importante tipo de biomoléculas. O texto exposto foi publicado no \emph{Journal of Membrane Science} \cite{meu3}.
	\item[Capítulo 7] Apresenta-se o último estudo desenvolvido  no âmbito deste trabalho de doutoramento, que se focou na otimização da separação pDNA/RNA por uma operação de UF. Pela primeira vez, o modelo desenvolvido no capítulo~\ref{chap:art3} é estendido para ser aplicável a um sistema de 4 componentes iónicos. Este melhoramento introduz a possibilidade de aferir as interações que as moléculas de pDNA e RNA produzem entre si durante processos de UF, e que se refletem nas permeações observadas. É obtida uma elevada seletividade do processo de UF/DF para efetuar a separação pDNA/RNA, com rendimentos comparáveis aos que se obtêm com um procedimento laboratorial alternativo que usa elevadas quantidades de solventes e de sais, e consequentemente com menor aplicabilidade a nível industrial. O texto exposto foi submetido para publicação no \emph{Journal of Membrane Science}.
	\item[Capítulo 8] Neste último capítulo são apresentadas as principais conclusões resultantes da realização do trabalho e as perspetivas de trabalho futuro.  
\end{description}
